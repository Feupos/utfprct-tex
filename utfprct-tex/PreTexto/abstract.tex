%%%% ABSTRACT
%%
%% Versão do resumo para idioma de divulgação internacional.

\begin{abstractutfpr}%% Ambiente abstractutfpr
The Notification-Oriented Paradigm (NOP) is a new approach to the construction
of computer systems. The NOP proposes computability by means of reactive and
decoupled entities model that interact by means of punctual notifications,
separating fact-executional from logic-causal computing. With this it is
possible to reduce or eliminate temporal and structural redundancies, common in
other programming paradigms, which can affect program performance. Still, the
intrinsic decoupling between NOP entities facilitates the construction of
concurrent and/or distributed systems. Moreover, the rule-oriented structure of
the NOP tends to ease development by allowing programming at a high level of
abstraction. NOP presents several materializations in software, being the
most mature technologically those that occur through frameworks, developed in
different programming languages. Among these frameworks, the one that presents
the highest degree of maturity and stability is the C++ Framework NOP 2.0.
However, the C++ Framework NOP 2.0 still has certain limitations, such as
excessive verbosity, low type flexibility and low algorithmic flexibility. In
this context, this work proposes the development of a new framework, named C++
Framework NOP 4.0, with the objective of removing the limitations present in the
C++ Framework NOP 2.0, as well as the imperfections of the C++ Framework NOP 3.0
that involves multithread/multicore, in order to improve the usability of the
NOP and its performance in this regard. The C++ Framework NOP 4.0 is developed
using generic techniques, by means of features added in the ISO C++11 C++14,
C++17 and C++20 and applying the test-driven development methodology. This
master's thesis presents the results obtained with the implementation of the C++
Framework NOP 4.0 through a set of relevant applications, such as the sensor
application, traffic light control application, and the algorithms Bitonic Sort
and Random Forest, both in a single thread and multithread/multicore
environment. These applications were executed and compared in terms of
performance against the same applications implemented with the C++ Framework NOP
2.0, Elixir/Erlang Framework NOP and also implementations in the Object-Oriented
Paradigm (OOP) in the C++ programming language and Procedural Paradigm (PP) in
the C programming language. As a result of these comparisons, the new C++
Framework NOP 4.0 proves to be superior to C++ Framework NOP 2.0 in both runtime
and memory consumption in the evaluated scenarios, besides presenting CPU
utilization levels comparable to the Framework Elixir/Erlang Framework
NOP multicore environment. Usability improvements are additionally
evaluated and attested by feedback from NOP developers.
\end{abstractutfpr}
