%%%% RESUMO
%%
%% Apresentação concisa dos pontos relevantes de um texto, fornecendo uma visão rápida e clara do conteúdo e das conclusões do trabalho.

\begin{resumoutfpr}%% Ambiente resumoutfpr
O Paradigma Orientado a Notificações (PON) é uma nova abordagem para a
construção de sistemas computacionais. O PON propõe a computação por meio de um
modelo de entidades reativas desacopladas que interagem por meio de notificações
pontuais, dentre as quais se divide e separa a computação facto-execucional da
computação lógico-causal. Com isso é possível reduzir ou eliminar redundâncias
temporais e estruturais, comuns em outros paradigmas de programação, que podem
afetar o desempenho dos programas. Ainda, o desacoplamento intrínseco entre as
entidades do PON facilita a construção de sistemas concorrentes e/ou
distribuídos. Além disso, a estrutura orientada a regras do PON tende a
facilitar o desenvolvimento por permitir programar em alto nível de abstração. O
PON apresenta várias materializações em \textit{software}, sendo as mais maduras
tecnologicamente aquelas que se dão por meio de \textit{frameworks},
desenvolvidos em diferentes linguagens de programação. Dentre estes
\textit{frameworks} o que apresenta o maior grau de maturidade e estabilidade é
o \textit{Framework} PON C++ 2.0. Entretanto, o \textit{Framework} PON C++ 2.0
ainda apresenta certas limitações, como excessiva verbosidade, baixa
flexibilidade de tipos e baixa flexibilidade algorítmica. Nesse contexto este
trabalho propõe o desenvolvimento de um novo \textit{framework}, denominado
\textit{Framework} PON C++ 4.0, com o objetivo de remover as limitações
presentes no \textit{Framework} PON C++ 2.0, bem como as imperfeições do
\textit{Framework} PON C++ 3.0  que envolve \textit{multithread/multicore}, de
forma a melhorar a usabilidade do PON e seu desempenho neste âmbito. O
\textit{Framework} PON C++ 4.0 é desenvolvido utilizando técnicas de programação
genérica, por meio de recursos adicionados nas versões do padrão ISO C++11,
C++14, C++17 e C++20, bem como aplicando o método de desenvolvimento orientado a
testes. Esta dissertação de mestrado apresenta os resultados obtidos com a
implementação do \textit{Framework} PON C++ 4.0 por meio de um conjunto de
aplicações pertinentes, tanto em ambiente \textit{single thread} quanto
\textit{multithread}/\textit{multicore}. Tais aplicações são um sistema de
sensores e uma aplicação de controle automatizado de tráfego, oriundos do grupo
de pesquisa, e dos algoritmos \textit{Bitonic Sort} e \textit{Random Forest}
oriundos da literatura. Tais aplicações foram executadas e comparadas em termos
de desempenho com as mesmas aplicações implementadas no \textit{Framework} PON
C++ 2.0, \textit{Framework} PON Elixir/Erlang e também implementações no
Paradigma Orientado a Objetos (POO) em linguagem de programação C++ e Paradigma
Procedimental (PP) em linguagem de programação C. Como resultado destas
comparações, o novo \textit{Framework} PON C++ 4.0 se mostra superior ao
\textit{Framework} PON C++ 2.0 tanto em tempo de execução como consumo de
memória nos cenários avaliados, além de apresentar balanceamento de carga
comparável aos do \textit{Framework} PON Elixir/Erlang em ambiente
\textit{multicore}. As melhorias na usabilidade são adicionalmente avaliadas e
atestadas pelo \textit{feedback} de desenvolvedores do PON.
\end{resumoutfpr}
