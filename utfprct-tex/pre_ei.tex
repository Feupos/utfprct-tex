\chapter{Prefácio}

Este trabalho foi realizado sob a forma de um Estudo Individual, composto da
realização de experimentos com o \textit{Framework} PON C++ 4.0, uma nova versão
de \textit{framework} para o Paradigma Orientado a Notificações (PON), que
apresentado no Seminário de Qualificação deste mesmo autor.

O objetivo deste Estudo individual é avaliar \textit{Framework} PON C++ 4.0 em
diversos cenários, afim de apresenta-los em um capítulo de resultados na
Dissertação que será subsequente desenvolvida. 

Para este propósito, foram desenvolvidos testes unitários e testes de
desempenho, incluindo as aplicações de sensores, Random Forest e Bitonic Sort,
que serão apresentadas no capítulo seguinte. Por meio destes testes é possível
analisar os tempos de execução, consumo de memória e uso de CPU do
\textit{Framework} PON C++ 4.0.